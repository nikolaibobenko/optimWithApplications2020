\documentclass[a4paper,11pt]{article}
\usepackage[french]{babel}
\usepackage[latin1]{inputenc}
%\usepackage{umlaut,amssymb,amsmath,amscd,a4,amsfonts}
\usepackage{amssymb,amsmath,amscd,a4,amsfonts,amsthm,mathrsfs}
%(a4 = 210 X 297 mm)
\hoffset -1in \voffset -1in \oddsidemargin 20mm \evensidemargin
\oddsidemargin \textwidth 170mm \topmargin 5mm \textheight 247mm

\newtheorem{theorem}{Theorem}
\newtheorem{lemma}{Lemma}

\theoremstyle{definition}
\newtheorem{exercise}{Exercise}

\DeclareMathOperator*{\argmin}{arg\,min}


\begin{document}

\pagestyle{headings}
\noindent UNIVERSITE DE GENEVE \hfill Section de Math�matiques\\
\noindent Facult\'e des sciences \hfill \\[-3mm]
\hrule

\large

\begin{center}
\textbf{Optimization with Application I \\ Exercise Sheet 5 - Discussed on 04.12.2020}
\end{center}
\hrule
\text{}\\[1cm]


\begin{exercise}
	For $y \in \mathbb{R}^N, X\in \mathbb{R}^{N\times P}$ and $\lambda > 0$, derive the formula to implement a relaxation algorithm to solve:
	\begin{itemize}
	 	\item Ridge Regression
	 	\[\argmin_\alpha \frac{1}{2}||y-X\alpha||_2^2 + \lambda ||\alpha||_2^2\]
	 	\item The Lasso
	 	\[\argmin_\alpha \frac{1}{2}||y-X\alpha||_2^2 + \lambda ||\alpha||_1\] 
	\end{itemize} 
\end{exercise}

\begin{exercise}
	
\end{exercise}



\end{document}
