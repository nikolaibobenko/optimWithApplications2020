\documentclass[a4paper,11pt]{article}
\usepackage[french]{babel}
\usepackage[latin1]{inputenc}
%\usepackage{umlaut,amssymb,amsmath,amscd,a4,amsfonts}
\usepackage{amssymb,amsmath,amscd,a4,amsfonts,amsthm}
%(a4 = 210 X 297 mm)
\hoffset -1in \voffset -1in \oddsidemargin 20mm \evensidemargin
\oddsidemargin \textwidth 170mm \topmargin 5mm \textheight 247mm

\newtheorem{theorem}{Theorem}
\newtheorem{lemma}{Lemma}

\theoremstyle{definition}
\newtheorem{exercise}{Exercise}

\begin{document}

\pagestyle{headings}
\noindent UNIVERSITE DE GENEVE \hfill Section de Math�matiques\\
\noindent Facult\'e des sciences \hfill \\[-3mm]
\hrule

\large

\begin{center}
\textbf{Optimization with Applications I \\ TP 2}
\end{center}
\hrule
\text{}\\[1cm]

\begin{exercise}
    Implement the Newton algorithm used to find roe minimum of a function. It should be a function with the signature
    \begin{verbatim}
        def function newton(f_der, f_der2, x_0, epsi=1.e-6, max_steps=1000):
                # your code here
            if not converged:
                raise Exception("does not converge")
            else:
                return (zero it converged to)
    \end{verbatim}
    Here \emph{f\_der} and \emph{f\_der2} should be the function's first and second derivatives calculated analytically.

    Test it on multiple functions with multiple starting values and find good examples of convergence and non-convergence.
\end{exercise}

\begin{exercise}
    Replace the input of the derivative in the algorithm by the first week's numerical derivative, so the new newton has the signature
    \begin{verbatim}
        def function num_newton(f_der, x_0, epsi=1.e-6, max_steps=1000):
                # your code here
            if not converged:
                raise Exception("does not converge")
            else:
                return (zero it converged to)
    \end{verbatim}
    Try to reuse the code from the first exercise.

    Can you find an example that converges for the analytical derivative but not for the numerical? 
\end{exercise}

\begin{exercise}
    Implement the golden section algorithm to find a minimum of a given unimodal function $f$ over the interval $[a,b]$.
    \begin{verbatim}
        def function golden_section(f, a, b, eps=1.e-6):
            # your code here
            return minimum
    \end{verbatim}
\end{exercise}

\begin{exercise}
    We want to perform linear regression like in Exercise $3$ on TD1. In order to produce nice pictures we will do this in dimension $2$.

    Pick some parameters $\alpha_0, \alpha_1$ and a variance $\sigma^2$. Draw $n = 50$ points $\{x_i\}_{i=1}^n$ at random from some interval you choose. Produce the values $y_i = \alpha_0 + \alpha_1 x_i + \epsilon_i$ where $\epsilon_i \sim \mathcal{N}(0, \sigma^2)$.

    Now we only treat $\{y_i\}$ as input and want to find the maximum likelihood $\alpha_0, \alpha_1, \sigma^2$. Find these values and plot the $x,y$ plot as well as the line defined by $\alpha_0, \alpha$.

    How does your result compare to the actual variables that created the data?
\end{exercise}

\bigskip

{\bf Important} : Every function and exercise must be tested. Plug in some values for which you know the correct answers and compare the output of your function. 

\end{document}
