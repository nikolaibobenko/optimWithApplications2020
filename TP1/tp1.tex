\documentclass[a4paper,11pt]{article}
\usepackage[french]{babel}
\usepackage[latin1]{inputenc}
%\usepackage{umlaut,amssymb,amsmath,amscd,a4,amsfonts}
\usepackage{amssymb,amsmath,amscd,a4,amsfonts,amsthm}
%(a4 = 210 X 297 mm)
\hoffset -1in \voffset -1in \oddsidemargin 20mm \evensidemargin
\oddsidemargin \textwidth 170mm \topmargin 5mm \textheight 247mm

\newtheorem{theorem}{Theorem}
\newtheorem{lemma}{Lemma}

\theoremstyle{definition}
\newtheorem{exercise}{Exercise}

\begin{document}

\pagestyle{headings}
\noindent UNIVERSITE DE GENEVE \hfill Section de Math�matiques\\
\noindent Facult\'e des sciences \hfill \\[-3mm]
\hrule

\large

\begin{center}
\textbf{Optimization with Applications I \\ TP 1}
\end{center}
\hrule
\text{}\\[1cm]


These exercises are meant to familiarize yourself with the development environment around python. Please install something like PyCharm and then solve the following.

\begin{exercise}[First function]
    Implement a function that given a value $x \in [1, \infty)$ returns $\sqrt{x - 1}$. Plot this function using the library matplotlib.
\end{exercise}

\begin{exercise}[Numerical derivative]
    Implement a function \emph{numerical\_derivative} that gets as input a function $f: \mathbb{R} \to \mathbb{R}$, a point $x_0 \in \mathbb{R}$ and the value $\epsilon > 0$ and returns the left a right difference quotients
    \[\frac{f(x_0 + \epsilon) - f(x_0)}{\epsilon}, \quad \frac{f(x_0 - \epsilon) - f(x_0)}{-\epsilon}.\]
    Test this with the function from Exercise 1 and compare to the actual value. 
\end{exercise}

\begin{exercise}[Numerical gradient]
    Write a function \emph{numerical\_gradient} that is given a multivariate function $f: \mathbb{R}^n \to \mathbb{R}$, a point $x_0 \in \mathbb{R}^n$ and the value $\epsilon > 0$ and returns the approximation to the gradient. You can use your function from Exercise 2.
\end{exercise}


\bigskip

{\bf IMPORTANT}: Every function and exercise must be tested. Plug in some values for which you know the correct answers and compare the output of your function.


\end{document}
