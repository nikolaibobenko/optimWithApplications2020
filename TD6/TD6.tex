\documentclass[a4paper,11pt]{article}
\usepackage[french]{babel}
\usepackage[latin1]{inputenc}
%\usepackage{umlaut,amssymb,amsmath,amscd,a4,amsfonts}
\usepackage{amssymb,amsmath,amscd,a4,amsfonts,amsthm,mathrsfs}
%(a4 = 210 X 297 mm)
\hoffset -1in \voffset -1in \oddsidemargin 20mm \evensidemargin
\oddsidemargin \textwidth 170mm \topmargin 5mm \textheight 247mm

\newtheorem{theorem}{Theorem}
\newtheorem{lemma}{Lemma}

\theoremstyle{definition}
\newtheorem{exercise}{Exercise}

\DeclareMathOperator*{\argmin}{arg\,min}


\begin{document}

\pagestyle{headings}
\noindent UNIVERSITE DE GENEVE \hfill Section de Math�matiques\\
\noindent Facult\'e des sciences \hfill \\[-3mm]
\hrule

\large

\begin{center}
\textbf{Optimization with Application I \\ Exercise Sheet 6 - Discussed on 11.12.2020}
\end{center}
\hrule
\text{}\\[1cm]

We will have a look at a classification problem. We are given a data set $(x_i, y_i)_{i=1}^n$ with $x_i \in \mathbb{R}^p$ and $y_i \in \{-1, 1\}$. These are points in $p$-dimensional space that each have a class $\pm 1$ associated with them. We will call the set of points in class 1 $A = \{x_i: \; y_i = 1\}$ and $B = \{x_i: \; y_i = -1\}$.

A hyperplane is given by a vector $w\in \mathbb{R}^p$ and an offset $b \in \mathbb{R}$ via $\{x: \; <x, w> - b = 0\}$.

\begin{exercise}
	Assume that there exists a hyperplane separating $A$ and $B$.
	\begin{enumerate}
		\item We want to find the hyperplane $H$ that has a maximum margin on both sides. That is we want to find the $w, b$ such that $\min_{x\in A} \text{dist}(x, H) = \min_{x \in B} \text{dist}(x, H)$ is maximal. Formulate this as a constrained optimization problem.
		\item Solve this problem using Lagrange multipliers. (You may need to expand on the cases we have considered, since your constraint will be an inequality.)
		\item How would you now classify a new point $x \in \mathbb{R}^p$?
	\end{enumerate}
\end{exercise}

\begin{exercise}
	Let $p = 2$ and $A \subset \mathbb{D}$, $B \subset \mathbb{C} \setminus \mathbb{D}$, where $\mathbb{D} = \{x: ||x||_2 \leq 1\}$. Find a mapping $\phi: \mathbb{R}^2 \to \mathbb{R}^3$ such that $\phi(A)$ is separable from $\phi(B)$ by a hyperplane.

	What changes in the formulas from Exercise 1 when considering such a mapping first?
\end{exercise}


\end{document}
